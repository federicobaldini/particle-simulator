\documentclass[a4paper,11pt]{article}
\usepackage[utf8]{inputenc}
\usepackage[T1]{fontenc}
\usepackage{graphicx}
\usepackage{amsmath}
\usepackage{hyperref}
\usepackage{array}
\usepackage{hyperref} 

\title{Simulazione ed analisi di particelle}
\author{Andrea Luiso, Elia Giulietti, Federico Baldini}
\date{2024}

\begin{document}

\maketitle
\newpage

\tableofcontents
\newpage

\section{Introduzione}

Questo progetto ha lo scopo di simulare eventi di collisione tra particelle, riproducendo il comportamento di particelle elementari in un ambiente controllato. Attraverso questa simulazione, il programma genera le proprietà fisiche delle particelle (ad esempio: angolo, quantità di moto e tipo) e calcola delle grandezze derivate, come energia e massa invariante. Il risultato della simulazione è analizzato mediante il framework \textbf{ROOT}, utilizzato per creare istogrammi e verificare la consistenza dei dati rispetto alle distribuzioni teoriche attese.

Questa simulazione include anche il decadimento di particelle risonanti, come \( K^* \), garantendo la conservazione della quantità di moto e la generazione dei relativi prodotti di decadimento. Lo scopo principale del progetto è studiare i picchi di risonanza, analizzando le masse invarianti e verificando la coerenza delle proprietà distribuite.

\clearpage

\section{Struttura del codice}

Il progetto è organizzato modularmente per favorire la riusabilità e la leggibilità del codice. Le principali classi implementate sono:

\begin{itemize}
    \item \textbf{ParticleType}: Classe base che rappresenta un tipo di particella generico. Contiene informazioni fondamentali come il nome, la massa e la carica. Questa classe fornisce un'implementazione comune per tutte le particelle generiche.
    \item \textbf{ResonanceType}: Estende la classe \texttt{ParticleType} aggiungendo funzionalità specifiche per le particelle risonanti, come la larghezza di decadimento (resonance width). Questo approccio sfrutta l'ereditarietà per evitare la duplicazione del codice e condividere proprietà comuni tra particelle generiche e risonanti.
    \item \textbf{Particle}: Classe che rappresenta una singola particella fisica. Contiene attributi come la quantità di moto e il tipo di particella. Fornisce metodi per calcolare proprietà derivate, come l'energia totale e la massa invariante, e include il metodo \texttt{Decay2Body} per simulare il decadimento di particelle risonanti.
\end{itemize}

Queste classi e la loro suddivisione consente di usare caratteristiche tipiche della programmazione ad oggetti, tra le quali:

\begin{itemize}
    \item \textbf{Ereditarietà}: Le proprietà comuni sono definite nella classe \texttt{ParticleType} e riutilizzate da \texttt{ResonanceType}.
    \item \textbf{Astrazione}: Le operazioni sulle particelle (ad esempio, il calcolo della massa invariante) sono incapsulate in metodi specifici della classe \texttt{Particle}, evitando duplicazioni nei calcoli.
    \item \textbf{Organizzazione modulare}: Le classi sono separate in file distinti, migliorando la manutenibilità del codice.
\end{itemize}

L'uso di questa struttura permette inoltre di mantenere il codice flessibile e scalabile, facilitando l'aggiunta di nuovi tipi di particelle o funzionalità future senza modificare il comportamento delle classi esistenti.

\clearpage

\section{Generazione dati}

Durante la simulazione sono stati generati 100.000 eventi di collisione, per un totale di 10.199.840 particelle. I tipi di particelle generate, insieme alle loro proporzioni teoriche e osservate, sono:

\begin{itemize}
    \item \textbf{Pion+} e \textbf{Pion-}: Ognuna con una probabilità teorica del 40\%. La simulazione ha prodotto rispettivamente il 39,7042\% e il 39,7188\% del totale, con una differenza minima rispetto al valore atteso.
    \item \textbf{Kaon+} e \textbf{Kaon-}: Con una probabilità teorica del 5\%, hanno registrato proporzioni osservate del 5,3869\% e 5,3846\%.
    \item \textbf{Proton+} e \textbf{Proton-}: Con una probabilità teorica del 4,5\%, le proporzioni osservate sono state rispettivamente del 4,4172\% e 4,4087\%.
    \item \textbf{K*}: Risonanza generata con una probabilità teorica dell'1\%, corrispondente allo 0,9796\% delle particelle totali.
\end{itemize}

Le proprietà cinematiche delle particelle, come quantità di moto, angoli azimutali (\( \phi \)) e polari (\( \theta \)), sono state generate casualmente. In particolare:

\begin{itemize}
    \item Gli angoli azimutali e polari seguono distribuzioni uniformi rispettivamente nell'intervallo \([0, 2\pi]\) e \([0, \pi]\). Le verifiche effettuate hanno mostrato un'eccellente accordo con le distribuzioni teoriche attese.
    \item La quantità di moto segue una distribuzione esponenziale con valore medio di 1 GeV/c. Il fit ha confermato una media osservata di \(0,9924 \pm 0,0005\) GeV/c, leggermente inferiore al valore teorico.
\end{itemize}

Gli eventi di collisione generati sono stati analizzati in base alla loro energia totale, alla quantità di moto trasversale ed alle masse invarianti. Complessivamente, sono stati registrati 515.281.596 ingressi per le masse invarianti, includendo combinazioni di tutte le coppie di particelle e specifici decadimenti del \( K^* \). Questa analisi permette di identificare picchi di risonanza e verificare la conservazione delle proprietà fisiche negli eventi simulati.

\clearpage

\section{Analisi}

In questa sezione vengono analizzati i risultati della simulazione, confrontando le distribuzioni osservate con i dati di input e verificando la congruenza delle proprietà delle particelle generate. Inoltre, viene descritto l'approccio seguito per estrarre il segnale della risonanza \( K^* \).

\subsection{Distribuzione delle particelle}

Le quantità dei diversi tipi di particelle sono state confrontate con le occorrenze attese. Come mostrato nella Tabella \ref{tab:abbondanze}, le occorrenze osservate sono in buon accordo con quelle teoriche, considerando gli errori statistici associati.

\begin{table}[h!]
    \centering
    \begin{tabular}{|c|c|c|}
        \hline
        \textbf{Specie} & \textbf{Occorrenze Osservate} & \textbf{Occorrenze Attese} \\
        \hline
        $\pi^+$ & $4\,049\,760 \pm 2\,012$ & $4\,000\,000$ \\
        \hline
        $\pi^-$ & $4\,051\,260 \pm 2\,013$ & $4\,000\,000$ \\
        \hline
        $K^+$ & $549\,452 \pm 741$ & $500\,000$ \\
        \hline
        $K^-$ & $549\,221 \pm 741$ & $500\,000$ \\
        \hline
        $p^+$ & $450\,544 \pm 671$ & $450\,000$ \\
        \hline
        $p^-$ & $449\,680 \pm 671$ & $450\,000$ \\
        \hline
        $K^*$ & $99\,920 \pm 316$ & $100\,000$ \\
        \hline
    \end{tabular}
    \caption{Distribuzione delle particelle: confronto tra occorrenze osservate ed attese.}
    \label{tab:abbondanze}
\end{table}

\subsection{Distribuzioni cinematiche}

Le distribuzioni degli angoli polari ($\theta$) e azimutali ($\phi$), insieme alla distribuzione del modulo dell'impulso, sono state analizzate tramite fit per verificare la loro congruenza con le distribuzioni teoriche attese.

\begin{table}[h!]
    \centering
    \begin{tabular}{|m{3cm}|>{\centering\arraybackslash}m{3cm}|>{\centering\arraybackslash}m{1.5cm}|>{\centering\arraybackslash}m{1.5cm}|>{\centering\arraybackslash}m{1.5cm}|}
        \hline
        \textbf{Distribuzione} & \textbf{Parametro del Fit} & $\chi^2$ & \textbf{DOF} & $\chi^2$/DOF \\
        \hline
        Angolo polare $\theta$ (fit costante) & $99\,998.9 \pm 31.6$ & $113.261$ & $99$ & $1.144$ \\
        \hline
        Angolo azimutale $\phi$ (fit costante) & $99\,999.2 \pm 31.6$ & $75.738$ & $99$ & $0.765$ \\
        \hline
        Modulo dell'impulso (fit esponenziale) & $0.9924 \pm 0.0005$ & $146.115$ & $88$ & $1.660$ \\
        \hline
    \end{tabular}
    \caption{Risultati dei fit per le distribuzioni degli angoli e del modulo dell'impulso.}
    \label{tab:distribuzioni}
\end{table}

Come evidenziato in Tabella \ref{tab:distribuzioni}, i parametri del fit per gli angoli polare e azimutale sono in accordo con una distribuzione uniforme. La distribuzione del modulo dell'impulso presenta una media leggermente inferiore al valore atteso di $1$ GeV/c, ma comunque compatibile entro gli errori statistici.

\subsection{Analisi della risonanza \( K^* \)}

Per estrarre il segnale della risonanza \( K^* \), è stata effettuata un'analisi delle masse invarianti delle coppie di particelle. In particolare, si è proceduto sottraendo l'istogramma delle combinazioni di carica concorde da quello delle combinazioni di carica discorde, al fine di ridurre il segnale di fondo e di isolare il picco della risonanza.

I risultati dei fit gaussiani sulle diverse distribuzioni sono riportati in Tabella \ref{tab:kstar}.

\begin{table}[h!]
    \centering
    \begin{tabular}{|m{2.5cm}|>{\centering\arraybackslash}m{2cm}|>{\centering\arraybackslash}m{2cm}|>{\centering\arraybackslash}m{2cm}|>{\centering\arraybackslash}m{1.5cm}|}
        \hline
        \textbf{Distribuzione e Fit} & \textbf{Media (GeV/$c^2$)} & \textbf{Sigma (GeV/$c^2$)} & \textbf{Ampiezza} & $\chi^2$/DOF \\
        \hline
        Massa invariante vere \( K^* \) & $0.8918 \pm 0.00016$ & $0.04987 \pm 0.00012$ & $2\,395 \pm 9$ & $1.050$ \\
        \hline
        Massa invariante da differenza carica discorde/concorde & $0.8945 \pm 0.0054$ & $0.0545 \pm 0.0054$ & $2\,270 \pm 196$ & $0.946$ \\
        \hline
        Massa invariante da differenza $\pi K$ carica discorde/concorde & $0.8939 \pm 0.0021$ & $0.0531 \pm 0.0022$ & $2\,297 \pm 80$ & $0.878$ \\
        \hline
    \end{tabular}
    \caption{Risultati dell'analisi della risonanza \( K^* \) tramite fit gaussiani.}
    \label{tab:kstar}
\end{table}

Le masse ottenute sono in ottimo accordo con il valore atteso per il \( K^* \) ($\approx 0.892$ GeV/$c^2$). L'approccio seguito ha permesso di ridurre il contributo del segnale di fondo e di evidenziare il segnale della risonanza, confermando la validità del metodo utilizzata.

\clearpage

\subsection{Analisi grafica}

Distribuzioni delle proprietà cinematiche e delle abbondanze delle particelle generate (in riferimento alla Figura \ref{fig:distribuzioni}), ogni box statistica mostra i parametri principali per il fit associato a ciascuna distribuzione.

\begin{figure}[h!]
    \centering
    \includegraphics[width=0.83\textwidth]{distributions.pdf}
    \caption{}
    \label{fig:distribuzioni}
\end{figure}


Analisi delle masse invarianti associate alla risonanza \( K^* \) (in riferimento alla Figura \ref{fig:kstar}):

\begin{figure}[h!]
    \centering
    \includegraphics[width=0.83\textwidth]{invariant_mass_distributions.pdf}
    \caption{}
    \label{fig:kstar}
\end{figure}

\end{document}